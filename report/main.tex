\documentclass[10pt,a4paper,twocolumn]{article}
\usepackage[utf8]{inputenc}
\usepackage[english]{babel}
\usepackage{amsmath}
\usepackage{amsfonts}
\usepackage{amssymb}
\usepackage{graphicx}
\usepackage{geometry}
\usepackage{fancyhdr}
\usepackage{listings}
\usepackage{xcolor}
\usepackage{hyperref}
\usepackage[most]{tcolorbox}
\usepackage{enumitem}
\usepackage{booktabs}
\usepackage{caption}
\usepackage{subcaption}
\usepackage{float}
\usepackage{titlesec}
\usepackage{microtype}
\usepackage{parskip}

% Geometry settings
\geometry{
    a4paper,
    top=2cm,
    bottom=2cm,
    left=1.5cm,
    right=1.5cm,
    headsep=0.5cm,
    footskip=1cm
}

\setlength{\headheight}{15pt}
\pagestyle{fancy}
\fancyhf{}
\rhead{Lab 01 - Parallelism and Concurrency}
\lhead{Software Architecture}
\cfoot{\thepage}

\pagenumbering{arabic}

% Code listing settings
\lstset{
    basicstyle=\ttfamily\scriptsize,
    keywordstyle=\color{blue}\bfseries,
    commentstyle=\color{green!60!black}\itshape,
    stringstyle=\color{red},
    showstringspaces=false,
    breaklines=true,
    breakatwhitespace=true,
    frame=leftline,
    framerule=2pt,
    rulecolor=\color{blue!30},
    backgroundcolor=\color{gray!5},
    numbers=left,
    numberstyle=\tiny\color{gray},
    stepnumber=1,
    numbersep=8pt,
    columns=flexible,
    aboveskip=\medskipamount,
    belowskip=\medskipamount,
    xleftmargin=15pt,
    language=Java
}

% Section formatting
\definecolor{darkblue}{RGB}{0,32,96}

% Cambiar todos los títulos y enlaces a darkblue

\titleformat{\section}
{\color{darkblue}\normalfont\large\bfseries}
{\thesection}{1em}{}

\titleformat{\subsection}
{\color{darkblue}\normalfont\normalsize\bfseries}
{\thesubsection}{1em}{}

\titleformat{\subsubsection}
{\color{darkblue}\normalfont\small\bfseries}
{\thesubsubsection}{1em}{}

% Custom boxes
\newtcolorbox{conceptbox}[1]{
    colback=blue!5!white,
    colframe=blue!75!black,
    colbacktitle=blue!80!black,
    coltitle=white,
    title={\textbf{#1}},
    fonttitle=\bfseries\small,
    boxrule=0.8pt,
    before skip=8pt,
    after skip=8pt,
    left=4pt,
    right=4pt,
    top=6pt,
    bottom=6pt,
    breakable
}

\newtcolorbox{resultbox}[1]{
    colback=green!5!white,
    colframe=green!75!black,
    colbacktitle=green!80!black,
    coltitle=white,
    title={\textbf{#1}},
    fonttitle=\bfseries\small,
    boxrule=0.8pt,
    before skip=8pt,
    after skip=8pt,
    left=4pt,
    right=4pt,
    top=6pt,
    bottom=6pt,
    breakable
}

\newtcolorbox{note}{
    colback=yellow!10!white,
    colframe=orange!75!black,
    colbacktitle=orange!80!black,
    coltitle=white,
    title={\textbf{Important Note}},
    fonttitle=\bfseries\small,
    boxrule=0.8pt,
    before skip=8pt,
    after skip=8pt,
    left=4pt,
    right=4pt,
    top=6pt,
    bottom=6pt,
    breakable
}

% List formatting
\setlist[itemize]{leftmargin=15pt, itemsep=2pt, parsep=0pt, topsep=5pt}
\setlist[enumerate]{leftmargin=15pt, itemsep=2pt, parsep=0pt, topsep=5pt}

% Caption formatting
\captionsetup{font=small, labelfont=bf, format=hang, indention=0pt, margin=10pt}

% Hyperref setup
\hypersetup{
    colorlinks=true,
    linkcolor=darkblue,
    urlcolor=darkblue,
    citecolor=darkblue,
    pdfborder={0 0 0}
}

% Title page
\title{\vspace{-1cm}
    \begin{center}
        \includegraphics[width=0.25\textwidth]{media/university_logo.png}
    \end{center}
    \vspace{1.5cm}
    \textbf{\Large Software Architecture Laboratory}\\
    \vspace{1cm}
    \textbf{\huge Laboratory No. 1}\\
    \textbf{\huge Parallelism and Concurrency}\\
    \vspace{1.5cm}
    \large Calculating $\pi$ Digits with Java 21 \& Spring Boot
}

\author{
    \vspace{2cm}
    \textbf{Students:}\\[0.05cm]
    Andersson David S\'{a}nchez M\'{e}ndez\\[0.3cm]
    Cristian Santiago Pedraza Rodr\'{i}guez\\[0.3cm]
    Elizabeth Correa Su\'{a}rez\\[0.3cm]
    Juan Sebastian Ortega Mu\~{n}oz\\[1.5cm]
    \textbf{Instructor:} Professor Javier Ivan Toquica Barrera\\[0.5cm]
    \textbf{Course:} Software Architecture (ARSW)\\[0.3cm]
    \textbf{Institution:} Escuela Colombiana de Ingenier\'{i}a Julio Garavito\\[2cm]
}

\date{January 30, 2026}

\begin{document}

\onecolumn
\maketitle
\thispagestyle{empty}
\newpage

\tableofcontents
\newpage

\twocolumn

% ============================================
\section{Objective}
% ============================================

This laboratory focuses on implementing parallelism concepts in a Spring Boot application that calculates hexadecimal digits of $\pi$ using the Bailey-Borwein-Plouffe (BBP) algorithm. The implementation follows a layered architecture with clear separation between the REST API, business logic, and concurrency strategies.

The specific goals include:


% ============================================
\section{Experiments and Results}
% ============================================

\subsection{API Documentation Evidence}
\begin{figure}[H]
    \centering
    \includegraphics[width=0.45\textwidth]{media/swagger-ui.png}
    \caption{Swagger UI documentation available at \texttt{/swagger-ui/index.html}.}
\end{figure}

\subsection{Testing and Coverage}
\begin{figure}[H]
    \centering
    \includegraphics[width=0.45\textwidth]{media/mvn-clean-test.png}
    \caption{Successful execution of \texttt{mvn clean test} (all tests pass).}
\end{figure}

\begin{figure}[H]
    \centering
    \includegraphics[width=0.45\textwidth]{media/mvn-clean-verify.png}
    \caption{Successful execution of \texttt{mvn clean verify} (coverage and quality checks passed).}
\end{figure}

\subsection{Benchmark Results by Team Member}

\subsubsection{Elizabeth -- AMD Ryzen AI 9 365}
\begin{figure}[H]
    \centering
    \includegraphics[width=0.45\textwidth]{media/elizabeth-benchmark-result.jpeg}
    \caption{Benchmark results on AMD Ryzen AI 9 365 (10C/20T, 2.00 GHz).}
\end{figure}

\subsubsection{Sebastian -- Intel i7-13620H}
\begin{figure}[H]
    \centering
    \includegraphics[width=0.45\textwidth]{media/sebastian-benchmark-result.jpeg}
    \caption{Benchmark results on Intel i7-13620H (10C/16T, 2.40 GHz).}
\end{figure}

\subsubsection{Andersson -- Intel i7-1255U}
\begin{figure}[H]
    \centering
    \includegraphics[width=0.45\textwidth]{media/andersson-benchmark-result.jpeg}
    \caption{Benchmark results on Intel i7-1255U (10C/12T, 1.70 GHz).}
\end{figure}

\subsubsection{Cristian -- Intel i5-1135G7}
\begin{figure}[H]
    \centering
    \includegraphics[width=0.45\textwidth]{media/cristian-benchmark-result.png}
    \caption{Benchmark results on Intel i5-1135G7 (4C/8T, 2.40 GHz).}
\end{figure}

\subsection{Summary of Experimental Results}
\begin{itemize}
    \item All implementations passed functional and quality tests, as evidenced by Maven and Swagger outputs.
    \item Benchmarks show that parallel strategies provide significant speedup, with optimal performance near the number of logical processors.
    \item AMD architecture achieved the highest speedup and best absolute performance; Intel hybrid and traditional architectures showed expected scaling and plateau/degradation at high thread counts.
    \item Results confirm Amdahl's Law: speedup is limited by the sequential fraction and hardware architecture.
\end{itemize}

\subsection{Strategy Pattern Implementation}

The concurrency layer implements the Strategy pattern through the \texttt{ParallelStrategy} interface:

\begin{lstlisting}[caption={ParallelStrategy Interface}]
public interface ParallelStrategy {
  String calculate(int start, 
                   int count, 
                   int threads);
  String name();
}
\end{lstlisting}

Two concrete strategies were implemented:

\begin{itemize}
    \item \textbf{SequentialStrategy}: Executes the calculation in a single thread, ignoring the \texttt{threads} parameter.
    \item \textbf{ThreadJoinStrategy}: Divides the workload among N threads and synchronizes results using \texttt{join()}.
\end{itemize}

\subsection{ThreadJoinStrategy Implementation}

The parallel strategy divides the digit range into segments distributed across threads:

\begin{lstlisting}[caption={Thread Work Division}]
int segmentSize = count / threads;
int remainder = count % threads;

for (int i = 0; i < threads; i++) {
  int segCount = segmentSize + 
    (i < remainder ? 1 : 0);
  // Create and start thread
  workers[i] = new Thread(() -> {
    results[index] = 
      calculateSegment(start, count);
  });
  workers[i].start();
}
\end{lstlisting}

The remainder distribution ensures even workload when \texttt{count} is not evenly divisible by \texttt{threads}. After all threads complete via \texttt{join()}, results are concatenated in order.

\subsection{REST API Endpoint}

The controller exposes a single endpoint with optional parallelism parameters:

\begin{lstlisting}[caption={API Endpoint}]
GET /api/v1/pi/digits
  ?start={int}     // Required, >= 0
  &count={int}     // Required, >= 1
  &threads={int}   // Optional, >= 1
  &strategy={str}  // Optional
\end{lstlisting}

Supported strategies: \texttt{sequential} (default) and \texttt{threads}.

\subsection{Service Layer Validation}

The \texttt{PiDigitsService} enforces business constraints:

\begin{table}[H]
    \centering
    \small
    \begin{tabular}{|l|l|}
        \hline
        \textbf{Parameter} & \textbf{Constraint} \\
        \hline
        \texttt{start} & $0 \leq$ start $\leq 100,000$ \\
        \hline
        \texttt{count} & $1 \leq$ count $\leq 10,000$ \\
        \hline
        \texttt{threads} & $1 \leq$ threads $\leq 200$ \\
        \hline
        \texttt{timeout} & 30 seconds max \\
        \hline
    \end{tabular}
    \caption{Service validation constraints}
    \label{tab:constraints}
\end{table}

% ============================================
\section{Theoretical Framework}
% ============================================

\subsection{Amdahl's Law}

The theoretical speedup when parallelizing a program is bounded by:

\begin{equation}
    S(n) = \frac{1}{(1 - P) + \frac{P}{n}}
\end{equation}

Where $P$ is the parallelizable fraction and $n$ is the number of threads. For the BBP algorithm, $P$ approaches 1.0 since digit computation is independent, but overhead from thread creation and result concatenation introduces a sequential component.

\subsection{Thread Synchronization with join()}

The \texttt{join()} method blocks the calling thread until the target thread terminates. This provides a simple synchronization mechanism:

\begin{lstlisting}[caption={Thread Synchronization}]
for (Thread worker : workers) {
  try {
    worker.join();
  } catch (InterruptedException e) {
    Thread.currentThread().interrupt();
    throw new RuntimeException(e);
  }
}
\end{lstlisting}

% ============================================
\section{Methodology}
% ============================================

\subsection{Test Environment}

Experiments were conducted using the following configuration:

\begin{table}[H]
    \centering
    \begin{tabular}{|l|l|}
        \hline
        \textbf{Component} & \textbf{Specification} \\
        \hline
        Java Version & OpenJDK 21 \\
        \hline
        Spring Boot & 3.x \\
        \hline
        Build Tool & Maven \\
        \hline
        Test Framework & JUnit 5 \\
        \hline
    \end{tabular}
    \caption{Development environment}
    \label{tab:env}
\end{table}

\subsection{Experiment Parameters}

Performance measurements were taken with:

\begin{itemize}
    \item \textbf{count}: Large values (1000+) for measurable execution times
    \item \textbf{start}: 0 (beginning of $\pi$ decimal expansion)
    \item \textbf{Thread configurations}: 1, N, 2N, 200 (where N = available processors)
\end{itemize}

% ============================================
\section{Results}
% ============================================

\subsection{Execution Time Comparison}

\begin{table}[H]
    \centering
    \small
    \begin{tabular}{|c|c|c|c|}
        \hline
        \textbf{Strategy} & \textbf{Threads} & \textbf{Time} & \textbf{Speedup} \\
        \hline
        Sequential & 1 & $T_{seq}$ & 1.00x \\
        \hline
        Threads & 1 & $\approx T_{seq}$ & $\approx$1.00x \\
        \hline
        Threads & N & $T_{N}$ & $T_{seq}/T_{N}$ \\
        \hline
        Threads & 2N & $T_{2N}$ & $T_{seq}/T_{2N}$ \\
        \hline
        Threads & 200 & $T_{200}$ & varies \\
        \hline
    \end{tabular}
    \caption{Execution times (N = available processors)}
    \label{tab:results}
\end{table}

\subsection{Observations}

\begin{resultbox}{Key Findings}
\begin{enumerate}
    \item Speedup increases with thread count up to the number of physical cores.
    \item Beyond core count, diminishing returns occur due to context switching overhead.
    \item With 200 threads, performance may degrade compared to optimal thread count.
    \item Results are deterministic: parallel output matches sequential output exactly.
\end{enumerate}
\end{resultbox}

% ============================================
\section{Analysis}
% ============================================

\subsection{Correctness Verification}

The implementation guarantees correctness through:

\begin{itemize}
    \item \textbf{Deterministic algorithm}: BBP produces identical results regardless of execution order.
    \item \textbf{Ordered concatenation}: Results are stored in indexed array and joined in order.
    \item \textbf{No shared mutable state}: Each thread writes to its own index in the results array.
\end{itemize}

\subsection{Thread Overhead}

Creating platform threads incurs overhead:

\begin{itemize}
    \item Thread object allocation
    \item OS thread scheduling
    \item Context switching between threads
    \item Synchronization at \texttt{join()} barrier
\end{itemize}

For small \texttt{count} values, this overhead may exceed the parallel computation benefit.

\subsection{Fallback Mechanism}

The service implements automatic fallback:

\begin{lstlisting}[caption={Fallback to Sequential}]
try {
  return threadJoinStrategy
    .calculate(start, count, threads);
} catch (Exception e) {
  logger.warn("Parallel failed");
  return sequentialStrategy
    .calculate(start, count, 1);
}
\end{lstlisting}

This ensures reliability when parallel execution fails.

% ============================================
\section{Conclusions}
% ============================================

\begin{enumerate}
    \item The Strategy pattern provides clean separation between sequential and parallel implementations, allowing runtime selection without code changes.
    
    \item The BBP algorithm is well-suited for parallelization due to its position-independent computation property.
    
    \item Optimal thread count typically matches the number of physical CPU cores. Exceeding this introduces overhead without proportional speedup.
    
    \item The \texttt{join()} synchronization mechanism provides simple and correct thread coordination for this use case.
    
    \item Input validation at the service layer prevents resource exhaustion (max 10,000 digits, 200 threads, 30s timeout).
\end{enumerate}

% ============================================
\section{References}
% ============================================

\begin{enumerate}
    \item Bailey, D., Borwein, P., Plouffe, S. (1997). On the Rapid Computation of Various Polylogarithmic Constants. \textit{Mathematics of Computation}, 66(218).
    
    \item Amdahl, G. M. (1967). Validity of the single processor approach to achieving large scale computing capabilities. \textit{AFIPS Conference Proceedings}.
    
    \item Oracle. Java 21 Documentation - Concurrency. \url{https://docs.oracle.com/en/java/javase/21/}
    
    \item Spring Framework. Spring Boot Reference Documentation. \url{https://spring.io/projects/spring-boot}
\end{enumerate}

\end{document}
